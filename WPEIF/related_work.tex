\section{Trabalhos Relacionados}
\label{sec:related-work}

Um sistema de autenticação 
seguro é construído sobre o controlador SDN em \citep{james2013HyperPass}.
O projeto RouteFlow  \citep{christian2010openflow} 
descreve um modelo de apresentação topológica da rede 
através de arquivos de configuração.
Apesar de serem relacionados, 
principalmente por tratarem de questões de gerenciamento,
nenhum desses trabalhos foca a representação da rede em modelo de grafo.

\citep{martin2010virtualizing} apresenta um modelo
de controlador em que a rede lógica e a topologia real da rede podem 
ser mapeadas como grafos.
O projeto Onix da Google \citep{teemu2010onix} 
apresenta uma abordagem SDN em que o 
controlador mantém uma visão global da rede em forma de grafo.
A NIB - \emph{Network Information Base}, um dos elementos fundamentais 
do projeto Onix \citep{teemu2010onix},
é um banco de dados distribuído, 
garantindo consistência e escalabidade à aplicação.
Ambos se relacionam com o presente trabalho,
porém não focam o tema de grafos.

Em \citep{ramya2012dynamic}, é apresentado um módulo de 
grafos com capacidade de atualização dinâmica e uma API pública,
implementado em C++ usando STL (\emph{Standard Template Library}).
Ele possibilita a adoção de algoritmos gerais que podem ser executados 
sobre o grafo, entretanto foca algoritmos de computação de rotas e atribuição de caminho.
O trabalho é geral, porém sem integração com controladores SDN na prática.

No geral, a representação em grafos proposta no presente trabalho 
também mantém dados sobre a rede, similar à NIB do Onix.
Como é possível registrar dados ``extras'' para cada \emph{Entity}, 
o módulo \emph{graph} vai além de manter informações topológicas.
Inevitavelmente, a necessidade de se buscar e monitorar dados 
específicos, principalmente em redes complexas, atrai o design de software 
de SDN para a direção de alguma forma de linguagem (DSL) que simplifique, 
organize, generalize e garanta eficiência na manipulação desses dados.
Esse é o exemplo do Frenetic \citep{Foster:2011:FNP:2034574.2034812} 
e do Pyretic \citep{Monsanto:2013:CSN:2482626.2482629}.

Enquanto a abordagem da DSL é baseada em atuação, execução,
conjunta com o controlador,
o NIB é baseada em um banco de dados distribuído,
com uma abordagem mais ampla e atacando problemas mais gerais
como persistência, concorrência, redundância, escalabilidade, etc.

O POX, segundo seu mantenedor, Murphy Mccauley, 
adotou o modelo de \emph{network fabric} \citep{martin2012fabric}. 
Esta arquitetura propõe uma divisão lógica em que a rede possui três elementos: 
\emph{hosts}, \emph{edge switchs} e \emph{core fabric}. 
O POX foca sua atuação no \emph{core fabric}, que, 
segundo essa arquitetura, lida apenas com transporte básico de pacotes.
Por isso, a abordagem do presente trabalho também lida com
controladores SDN relacionados com o \emph{core fabric}, 
porém incluindo uma visão dos \emph{hosts} e, se necessário, dos \emph{edge switchs}.
Uma generalização para a Internet dessa arquitetura pode ser 
vista em \citep{barath2012software}. 
