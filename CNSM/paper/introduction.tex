\section{Introduction}
\label{sec:introduction}

%Software Defined Networks (SDN) decouple the data plane from control plane.
%\citep{guedes2012redes}.
Software environments designed to provide SDN applications
are named SDN controllers.
They have also been called network operating systems because they provide a layer
that
isolates and control the access to the physical network elements, providing
a standardized interface to them. That interface may take different forms,
like events in NOX~\cite{nox}, an internal database in Onix~\cite{teemu2010onix},
a predicate-rule based reasoning system~\cite{fml2009}, 
or a query language in Frenetic~\cite{Foster:2011:FNP:2034574.2034812},
for example.

No matter the API, 
almost all applications of Software Defined Networks (SDN)
need a topological view of the network; in
fact, that global view is one of the key aspects of that
paradigm~\citep{martin2010virtualizing}.
Graphs model the network topology in a direct, natural and precise way,
so describing the network using a graph has become a common practice in many works,
protocols and network software, including those on
SDN~\citep{ramya2012dynamic}.
In this sense, a graph should be a basic resource of an SDN controller,
providing the
network representation, the access and the control of the network elements in
a single structure. Thus, a graph module can be of multiple uses in this kind of
software.

%A controller is composed of many interconnected modules and can interact with external systems.
A graph model would be useful for both the internal modules of the
SDN controller as for the applications using the controller.
They need topological information or just access to the network data,
and the search for and access to the network
entities can be provided by the graph module. 
It may have the capacity to notify state changes, be it through events,
callbacks or other notification mechanisms.

Moreover, in practice, many SDN applications use graph algorithms to obtain
informations that affect the control of the underlying network, such as
Shortest Path, Minimum Spanning Tree, Graph Coloring, etc.
The graph module store this kind of representation directly and execute any
of those algorithms internally,
making the results available for other software modules,
with no need for them to repeat the
computation, and for the application developer to re-implement 
complex data structures and algorithms.
A graph module can be implemented for that goal, avoiding new dependencies
among different modules.


The interactions with the graph module can be encapsulated and 
defined as a semantically stable and standardized interface. Its implementation
can be modified to adjust to different systems, using different resources like
local memory, remote databases, centralized or distributed processing,
concurrency control, 
parallelism, persistence, performance and other relevant characteristics.

%Considering the importance of this field on network operating systems, including
%the management software and applications, 
%this paper aims to exemplify and analyze the utilization of graph modules on SDN development.
%The controller used in this work is POX. 
%
%As other SDN controllers, POX has no network representation built on graph.


This paper presents a
description of a new kind of abstraction for network management,
integrating elements like automated fault detection and provision for dynamic
graphs,
a real implementation on a system using OpenFlow~\citep{nick2008openflow},
and its experimental validation.

The remainder of this paper provides, first, a description of the POX controller
and the design, properties and project decisions that lead to the graph model
implementation.
After that we show some experiments and results obtained in a network simulation
environment, in this case, Mininet~\citep{lantz2010network}.
Finally, we discuss some paths for future work and a brief conclusion. 
