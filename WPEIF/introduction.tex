\section{Introdução}
\label{sec:introduction}

Redes definidas por software (SDN) separam o plano de dados do plano 
de controle \citep{guedes2012redes}.
Softwares destinados para elaboração de aplicações SDN, 
como o NOX, o POX, o Onix, são chamados de controladores SDN.
Eles também são considerados sistemas operacionais de rede, pois 
formam uma camada que separa e fornece interface padronizada aos recursos da rede.
Protocolos padronizados como OpenFlow viabilizam as SDN.

As características de um software para SDN podem variar de acordo com o domínio da aplicação. 
Porém, uma característica em comum em quase todas as aplicações em SDN é a necessidade de 
uma ``visão'' topológica da rede.
Grafos representam a topologia de uma rede de forma direta, natural e precisa. 
Por isso, a descrição da rede em forma de grafos é um elemento comum em diversos
dos principais trabalhos, protocolos e softwares de rede, 
inclusive os elaborados para SDN,
como \citep{martin2010virtualizing}, \emph{Network Information Base}, \citep{ramya2012dynamic}.

Nesse sentido, o grafo é um recurso básico de um controlador SDN
que fornece a representação da rede.
Por isso, um módulo de grafo parece ser fundamental nesses tipos de software.
Em analogia, um módulo de grafo em um sistema operacional de rede é 
similar ao módulo de sistema de arquivos e outros módulos típicos 
em um sistema operacional clássico.
Contudo, diferentemente dos sistemas operacionais que já estão 
razoavelmente solidificados depois de anos de pesquisas e desenvolvimento,
os sistemas operacionais de rede, especialmente no modelo de SDN, 
ainda possuem vários temas específicos e carentes de pesquisa.
A representação da rede, o acesso ou controle de seus elementos
é um desse temas.

Um controlador é composto de diversos módulos que se interligam, 
podendo interagir com outros sistemas externos, 
inclusive legados. 
Um módulo de grafo é útil tanto para os módulos internos do controlador SDN,
quanto as aplicações elaboradas sobre o controlador.
Ambos, podem necessitar de informações da topologia ou 
simplesmente ter acesso aos dados, receber eventos e/ou
executar métodos sobre os objetos de entidades da rede, 
independente de serem entidades lógicas ou físicas.
Em termos de design e arquitetura do sistema operacional de rede, 
o acesso e a pesquisa às entidades de rede podem ser fornecidos 
por meio do módulo de grafo,
que também deve possuir a capacidade de notificar mudanças de estado, 
via eventos, \emph{callback} ou outros mecanismos de notificação.
Como a interação com o módulo de grafos pode ser encapsulada e abstraída 
em uma interface semanticamente estável e padronizada,
sua implementação pode variar para se adaptar a diferentes tipos de sistema, 
contando com diferentes recursos como o 
uso de memória local ou de banco de dados remoto, 
processamento centralizado ou distribuído, 
controle de concorrência, paralelismo, persistência, desempenho e 
vários outras características relevantes.

Além disto, na prática, vários softwares em SDN,
principalmente os de gerenciamento, utilizam informações
obtidas de algoritmos em grafo, como menor caminho,
árvore geradora mínima, coloração, etc.
O módulo de grafo pode executar ou registrar esse tipo de computação,
tornando disponível para qualquer outro software os resultados já processados,
sem que eles precisem repetir computação já executada e nem 
implementar algoritmos específicos que podem ser mantido no módulo de grafos.
Isso pode ser realizado sem criar dependência entre diferentes módulos,
bastando que sejam adotados os recursos do módulo básico de grafo.

Considerando-se a importância do tema para o desenvolvimento de 
sistemas operacionais de rede, 
incluindo os softwares de controle e os aplicativos,
o objetivo deste trabalho é exemplificar e analisar o uso de um módulo de grafos 
para o desenvolvimento de SDN.
O controlador usado na pesquisa é o POX,
pois, assim como outros softwares controladores SDN,
ainda não fornece esse tipo de recurso de forma ampla e geral.

Este artigo apresenta as seguintes contribuições. 
Primeiro, descrevemos um sistema para o gerenciamento de redes. 
As redes são representadas por grafos. 
Os grafos são dinâmicos, em tempo real e contêm estatísticas da rede. 
Este sistema é compatível com o padrão OpenFlow, um dos protocolos para Rede Definidas por Software. 
Depois, validamos o nosso sistema com experimentos. 
O sistema proposto permite a identificação dos elementos de rede e também detecção de falhas.

Este trabalho apresenta uma comparação com outros trabalhos correlatos. 
Em sequência, são descritas características de design relacionadas com o POX 
e as decisões de projeto relacionados com a implementação do módulo de grafo. 
Também são apresentados experimentos e resultados 
obtidos em um ambiente de simulação em redes, 
no caso, o Mininet \citep{lantz2010network}.
Por fim, são discutidos os aspectos futuros do trabalho e uma breve conclusão. 
