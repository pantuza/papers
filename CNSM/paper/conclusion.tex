\section{Conclusion}
\label{sec:conclusion}
This paper showed the use of graphs in the Software Defined Networking context.
A real time graph of the network meets one of the mainly advantages
expected from SDN by
decoupling the control plane from data plane, which is to 
achieve a global view of the network.
It is noted that the system keeps a reliable, consistent and 
dynamic representation of the real network, facilitating the management
tasks in a Software Defined Network.

As a future work we plan to build an on-line graph visualizer 
that interacts with the network administrator and shows, in 
an easier way, the entire network operation.
Various popular graph algorithms should be provided by the graph module. 
Those should be implemented in the future releases of the system.

One element of concern is reliability:
just like POX by itself, if the controller process is terminated, then the
entire graph is lost.
For that, a distributed graph database can be used to store
the network graph in a persistent, reliable and fault tolerant
structure.

Finally, the graph abstraction has been identified in other network
scenarios, like cloud computing. OpenStack, one of the most popular systems
for cloud management/virtualization orchestration, includes a network
topology view in one of its modules, Neutron~\cite{openstacksite}. That
abstraction has already been combined with SDN to implement isolated
multi-tenant networks~\cite{lcn2013}, it might be interesting to consider
how they could be further combined to build a unified, shared view.
