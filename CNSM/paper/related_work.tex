\section{Related work}
\label{sec:related-work}


The idea of representing the network as a graph was mentioned by Casado
\emph{et al.} in one of the first SDN papers~\cite{martin2010virtualizing}. 
However, there were no details about their implementation. In a later work,
SDN was used to implement different network topologies in a datacenter
scenario, but that was not done by implementing a graph abstraction inside
the controller~\cite{ripcord}.

Raghavendra \emph{et al.} presented a graph module with dynamic update
capabilities and a public API that could be extended to include different
graph algorithms~\cite{ramya2012dynamic}.
Although the work was aimed at SDN/cloud scenarios, there was
no actual integration with any SDN controller, which was the major
focus or the present work.

The Onix controller~\citep{teemu2010onix} was designed around the
concept of a network information base (NIB). That base keeps a global
view of the network in a form similar to SNMP's MIB. However, the
representation of the graph is achieved by indexing an element's entry in
relation to its neighbors, not directly.

DSLs (domain specific languages) are presented by 
Frenetic \citep{Foster:2011:FNP:2034574.2034812} and 
Pyretic \citep{Monsanto:2013:CSN:2482626.2482629} as good
solutions for the network data retrieval problem. Neither of them
export the network graph as a first order element, but it can be built
externally based on the information available.

