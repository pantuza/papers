\section{O design do POX}
\label{sec:design}

Esta seção descreve as características mínimas e necessárias do POX 
para atingir o objetivo de elaboração do módulo de grafos.

O POX é um arcabouço para a elaboração de softwares de SDN.
Ele é baseado em componentes.
Os componentes arquiteturais do POX são chamados de módulos.
É possível carregar módulos tanto de forma estática (via {\it import})
quanto dinâmica, durante a execução (via {\it eval}). 
Cada módulo, representado por sua classe, 
se publica ao \emph{core} através de um método obrigatório chamado \emph{launch}.

O \emph{OpenFlow} \citep{nick2008openflow} é um protocolo de comunicação
que possibilita o acesso do plano de encaminhamento de um \emph{switch} 
ou roteador.
Esse acesso é feito através de uma interface padronizada para esses
dispositivos de rede.
Essa comunicação permite experimentar novos protocolos facilitando 
o desenvolvimento de pesquisas em redes.
O POX, um controlador SDN, implementa um componente para comunicar com 
dispositivos de rede que estão habilitados a utilizar o protocolo 
\emph{OpenFlow}. 
Esse componente chamado \emph{openflow}, é responsável por enviar 
comandos e receber mensagens dos equipamentos de rede. 
Para o núcleo do controlador essas mensagens e comandos ocorrem na 
forma de eventos.

O \emph{core} (núcleo) define o funcionamento do POX. 
Uma de suas principais funções é proporcionar a comunicação entre os módulos.
O principal mecanismo de comunicação é o envio de eventos.
O modelo de eventos torna o design mais flexível e com menor acoplamento.
O \emph{core} oferece um canal de eventos no modelo ``push'',
ou seja, ele recebe o evento dos \emph{publishers} e envia para os \emph{subscribers}.
Via \emph{core}, cada módulo pode publicar novos canais de eventos 
ou se inscrever em canais de eventos.
O próprio \emph{core} publica alguns canais de eventos, 
como os relacionados com a carga de módulos.

O POX contém alguns módulos de descoberta que 
publicam eventos que se relacionam com a topologia da rede.
Esses módulos foram criados em contextos diferentes. 
Um módulo específico, o \emph{topology},
é responsável por mapear a topologia da rede. 
Utilizando os recursos do \emph{core}, 
ele atua como uma espécie de canal de eventos canônico,
conectando vários \emph{publishers} diferentes com 
os diversos \emph{subscribers}, 
sem que um necessite conhecer o outro diretamente. 
Isso o torna um ponto central genérico
para os diferentes módulos de descoberta e controle da rede 
que podem ser adicionados ou removidos, 
dependendo do caso concreto,
sem afetar os módulos de gerenciamento e visualização da rede.

Um grafo que represente globalmente a rede 
deve contar com a integração do \emph{topology} com os
demais módulos relacionados. 
Ou seja, o grafo proposto pelo trabalho 
é uma representação da rede em tempo 
real e atualizado dinamicamente em função das informações notificadas 
pelo \emph{topology}. 

O \emph{topology} publica dois tipos de eventos:
entrada ({\it join}) e saída ({\it leave}) 
de entidades ({\it entity}).
A entidade é uma classe abstrata que pode ser
especializada em switch e host, por exemplo.
Os diversos módulos notificam os eventos ao \emph{topology},
via chamada direta de métodos \emph{addEntity} e \emph{removeEntity}. 
O topology, no papel de  {\it publisher}, notifica seus {\it subscribers}
por meio do mecanismo de eventos padrão do POX.

Contudo, tanto o \emph{topology} quanto diversos outros módulos relevantes
necessitaram de novos recursos, como novas classes, 
ajustes de eventos, informações e estruturas de dados adicionais, etc.
Por motivo de compatibilidade, durante os ajustes das classes já existentes no POX,
tentou-se afetar o mínimo possível os programas, principalmente as interfaces.
Os módulos relevantes para atingir os objetivos do presente trabalho são:

\begin{itemize}

\item{\emph{topology}}: 
Canal de eventos canônico que generaliza os eventos relacionados a topologia da SDN.
Ele determina uma chave única para cada entidade de rede no contexto do POX,
unificando a identificação entre diferentes módulos.
Modificado para incluir a entidade, e seus respectivos eventos, 
relacionados com \emph{links},
além do evento não implementados \emph{HostEvent}.

\item{\emph{host\_tracker}}: 
Seu objetivo é buscar e rastrear informações sobre os {\it hosts} dentro da rede.
Minimamente modificado, principalmente para tratar eventos de DHCP.

\item{\emph{openflow.topology}}: 
Um controle de topologia específico para switches OpenFlow.
Ele é integrado com o módulo \emph{topology} do POX.
Também foi modificado para incluir a notificação de eventos relacionados com \emph{links}.

\item{\emph{openflow.discovery}}: 
Identifica links entre switches OpenFlow por meio do protocolo LLDP.

\item{\emph{misc.dhcpd}}:
Servidor de DHCP para configuração dinâmica da rede IP.

\end{itemize}

Outras questões relacionadas como, desempenho, estabilidade,
distribuição, balanceamento de carga, redundância, 
escalabilidade também vão além tanto do objetivo quanto da
implementação do POX ou de seus módulos.
É fundamental destacar que o POX, 
assim como diversos outros softwares de SDN, 
tem objetivos relacionados com a pesquisa científica, 
incluindo exemplos de conceitos e aplicações.
Seu objetivo e o estágio atual não é de software 
capaz de atuar em ambiente complexos ou de produção,
principalmente no qual a estabilidade é um fator crítico.
